%-------------------------------------------------------------------------------
%	SECTION TITLE
%-------------------------------------------------------------------------------
\cvsection{Experience}


%-------------------------------------------------------------------------------
%	CONTENT
%-------------------------------------------------------------------------------
\begin{cventries}

%---------------------------------------------------------
  \cventry
    {데브옵스 엔지니어 (인턴)} % Job title
    {당근페이 주식회사} % Organization
    {대한민국, 서울} % Location
    {2023.06 - 2023.08} % Date(s)
    {
      \begin{cvitems} % Description(s) of tasks/responsibilities
        \item {당근페이의 배포 생산성을 높이기 위한 데브옵스 도구 개발.}
        \item {신규 서비스의 성능 테스트 수행.}
        \item {핀테크 데브옵스/SRE 업무 수행.}
      \end{cvitems}
    }

%---------------------------------------------------------
  \cventry
    {백엔드 엔지니어 (계약)} % Job title
    {네이버 주식회사} % Organization
    {대한민국, 서울} % Location
    {2023.02 - 2023.06} % Date(s)
    {
      \begin{cvitems} % Description(s) of tasks/responsibilities
        \item {실시간 협업 애플리케이션 위한 오픈소스 문서 저장소 "Yorkie"의 프로덕션 환경 지원을 위한 Sharded 클러스터 모드 디자인 및 개발.}
        \item {Kubernetes와 Istio를 구현체로 Sharded 클러스터 모드 구현 및 Helm Chart로 패키징/배포.}
        \item {Yorkie의 프로덕션 환경 지원을 위해 Yorkie 서버의 master/slave 기반 GC, Document 삭제 API 등을 구현.}        
        \item {Terraform, ArgoCD, Prometheus, Grafana, Loki를 이용하여 Yorkie SaaS 데브옵스 시스템을 재구축.}
        \item {새로운 CRDT 자료구조 방식인 yorkie.tree를 Go SDK에 포팅.}
      \end{cvitems}
    }

%---------------------------------------------------------
  \cventry
    {오픈 프런티어 개발자 \& 멘토} % Job title
    {오픈 프런티어 (정보통신산업진흥원 주관 및 과학기술정보통신부 주최)} % Organization
    {Seoul, S.Korea} % Location
    {Mar. 2023 - Nov. 2023} % Date(s)
    {
      \begin{cvitems} % Description(s) of tasks/responsibilities
        \item {2023년 오픈 프런티어 개발자로 Yorkie 프로젝트 및 기타 오픈소스 활동 수행.}
        \item {2023년 오픈소스 플레이그라운드의 Yorkie 프로젝트 멘토로 활동.}
      \end{cvitems}
    }

%---------------------------------------------------------
  \cventry
    {소프트웨어 엔지니어} % Job title
    {소프트웨어 마에스트로 (정보통신기획평가원 주관 및 과학기술정보통신부 주최)} % Organization
    {대한민국, 서울} % Location
    {2022.03 - 2022.11} % Date(s)
    {
      \begin{cvitems} % Description(s) of tasks/responsibilities
        \item {시차가 존재하는 원격 팀을 위한 화이트보드 + 비디오 로그 기반 비동기 협업 소프트웨어 "Asyncrum" 기획 및 개발.}
        \item {Asyncrum 전체 소프트웨어 개발 프로세스 수립 및 짧은 개발 사이클을 가져가기 위한 스크럼 방식/도구 도입}.
        \item {백엔드 아키텍처 설계 및 AWS 클라우드 인프라 구축, 이후 v1.0부터 v1.6까지의 아키텍처 개선.}
        \item {Jenkins와 ArgoCD를 통한 백엔드 CI/CD 파이프라인 구축.}
        \item {화이트보드 오픈소스 tldraw와 동시 편집 오픈소스 Yorkie를 결합하여 실시간 협업 화이트보드 프로토타이핑 및 구현.}     
        \item {실시간 협업 화이트보드의 지연시간을 throttle 방식을 통해 대략 70\% 개선.}    
      \end{cvitems}
    }

%---------------------------------------------------------
  \cventry
    {소프트웨어 엔지니어} % Job title
    {PH2Y (세종대학교)} % Organization
    {대한민국, 서울} % Location
    {2021.07 - 2022.06} % Date(s)
    {
      \begin{cvitems} % Description(s) of tasks/responsibilities
        \item {안드로이드 지도 기반 반려동물 정보 공유 플랫폼 "집사의 노트" 서비스 기획 및 개발.}
        \item {자바 이미지 스케일링 라이브러리인 imgscalr를 활용한 백엔드 이미지 처리 서비스 구현 및 이미지 데이터 14배 \~ 72배 최적화.}
        \item {안드로이드 비디오 스트리밍을 위해 HTTP range request를 활용하여 비디오 스트리밍 서비스 구현.}
      \end{cvitems}
    }

%---------------------------------------------------------
  % \cventry
  %   {Software Engineer} % Job title
  %   {ShitOne Corp.} % Organization
  %   {Seoul, S.Korea} % Location
  %   {Dec. 2011 - Feb. 2012} % Date(s)
  %   {
  %     \begin{cvitems} % Description(s) of tasks/responsibilities
  %       \item {Developed a proxy drive smartphone application which connects proxy driver and customer.}
  %       \item {Implemented overall Android application logic and wrote API server for community service, along with lead engineer who designed bidding protocol on raw socket and implemented API server for bidding.}
  %     \end{cvitems}
  %   }

%---------------------------------------------------------
\end{cventries}
