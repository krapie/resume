%-------------------------------------------------------------------------------
%	SECTION TITLE
%-------------------------------------------------------------------------------
\cvsection{Extracurricular Activity}


%-------------------------------------------------------------------------------
%	CONTENT
%-------------------------------------------------------------------------------
\begin{cventries}

%---------------------------------------------------------
  \cventry
    {메인테이너} % Affiliation/role
    {Yorkie} % Organization/group
    {Yorkie Team} % Location
    {2023.03 - } % Date(s)
    {
      \begin{cvitems} % Description(s) of experience/contributions/knowledge
        \item {Yorkie는 실시간 협업 애플리케이션을 개발하기 위한 오픈소스 문서 저장소.}
        \item {Yorkie 프로젝트의 메인테이너 및 개발자, 멘토로 활동 중.}
        \item {Yorkie Sharded Cluster Mode 디자인 및 개발.}
        \item {Yorkie SaaS 인프라 환경 구축 및 관리.}
        \item {Yorkie 서버 및 클라이언트의 RPC 프로토콜 개선.}
        \item {Yorkie 서버의 성능 개선 및 안정화.}
        \item {Yorkie 실시간 협업 화이트보드 예제 개발.}
        \item {Yorkie 프로젝트 전반적인 기능, 최적화, 이슈 등을 제안.}
      \end{cvitems}
    }

%---------------------------------------------------------
  \cventry
  {멤버} % Affiliation/role
  {Kubernetes} % Organization/group
  {kubernetes-sig-docs} % Location
  {2023.04 - } % Date(s)
  {
    \begin{cvitems} % Description(s) of experience/contributions/knowledge
      \item {Kubernetes는 컨테이너화된 워크로드와 서비스를 관리하기 위한 이식성이 있고, 확장 가능한 오픈소스 플랫폼.}
      \item {kubernetes-sig-docs 한글화 팀에서 번역 및 문서 업데이트를 진행.}
      \item {kubernetes v1.26 버전 기준 8건의 신규 문서 번역 진행 및 outdated files 업데이트 진행.}
    \end{cvitems}
  }

%---------------------------------------------------------
  \cventry
    {멤버} % Affiliation/role
    {Istio} % Organization/group
    {Istio Community} % Location
    {2023.04 - } % Date(s)
    {
      \begin{cvitems} % Description(s) of experience/contributions/knowledge
        \item {Istio는 서비스 메시를 위한 오픈소스 플랫폼.}
        \item {Istio 커뮤니티의 멤버로 활동.}
        \item {istio.io 문서에 누락된 gRPC probe health checking과 관련하여 문서 업데이트를 수행.}
      \end{cvitems}
    }

%---------------------------------------------------------
\cventry
    {오픈 프런티어 개발자 \& 멘토} % Affiliation/role
    {오픈 프런티어 (정보통신산업진흥원 주관 및 과학기술정보통신부 주최)} % Organization/group
    {대한민국, 서울} % Location
    {2023.05 - 2023.11} % Date(s)
    {
      \begin{cvitems} % Description(s) of experience/contributions/knowledge
        \item {2023년 오픈 프런티어 개발자로 Yorkie 프로젝트 개발 및 확산 활동 수행.}
        \item {2023년 오픈소스 컨트리뷰션 아카데미의 Yorkie 프로젝트 멘토 수행.}
      \end{cvitems}
    }
    
%---------------------------------------------------------
  \cventry
    {소프트웨어 엔지니어} % Affiliation/role
    {소프트웨어 마에스트로 (정보통신기획평가원 주관 및 과학기술정보통신부 주최)} % Organization
    {대한민국, 서울} % Location
    {2022.03 - 2022.11} % Date(s)
    {
      \begin{cvitems} % Description(s) of experience/contributions/knowledge
        \item {시차가 존재하는 원격 팀을 위한 화이트보드 + 비디오 로그 기반 비동기 협업 소프트웨어 "Asyncrum" 기획 및 개발.}
        \item {Asyncrum 전체 소프트웨어 개발 프로세스 수립 및 짧은 개발 사이클을 가져가기 위한 스크럼 방식/도구 도입}.
        \item {백엔드 아키텍처 설계 및 AWS 클라우드 인프라 구축, 이후 v1.0부터 v1.6까지의 아키텍처 개선.}
        \item {Jenkins와 ArgoCD를 통한 백엔드 CI/CD 파이프라인 구축.}
        \item {화이트보드 오픈소스 tldraw와 동시 편집 오픈소스 Yorkie를 결합하여 실시간 협업 화이트보드 프로토타이핑 및 구현.}     
        \item {실시간 협업 화이트보드의 지연시간을 throttle 방식을 통해 대략 70\% 개선.}    
      \end{cvitems}
    }

%---------------------------------------------------------
  \cventry
    {소프트웨어 엔지니어} % Affiliation/role
    {PH2Y (세종대학교)} % Organization
    {대한민국, 서울} % Location
    {2021.07 - 2022.06} % Date(s)
    {
      \begin{cvitems} % Description(s) of experience/contributions/knowledge
        \item {안드로이드 지도 기반 반려동물 정보 공유 플랫폼 "집사의 노트" 서비스 기획 및 개발.}
        \item {자바 이미지 스케일링 라이브러리인 imgscalr를 활용한 백엔드 이미지 처리 서비스 구현 및 이미지 데이터 14배 \~ 72배 최적화.}
        \item {안드로이드 비디오 스트리밍을 위해 HTTP range request를 활용하여 비디오 스트리밍 서비스 구현.}
      \end{cvitems}
    }


%---------------------------------------------------------
\end{cventries}
