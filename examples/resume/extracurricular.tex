%-------------------------------------------------------------------------------
%	SECTION TITLE
%-------------------------------------------------------------------------------
\cvsection{Extracurricular Activity}


%-------------------------------------------------------------------------------
%	CONTENT
%-------------------------------------------------------------------------------
\begin{cventries}

%---------------------------------------------------------
  \cventry
    {메인테이너} % Affiliation/role
    {Yorkie} % Organization/group
    {Yorkie Team} % Location
    {2023.03 - } % Date(s)
    {
      \begin{cvitems} % Description(s) of experience/contributions/knowledge
        \item {Yorkie는 실시간 협업 애플리케이션을 개발하기 위한 오픈소스 문서 저장소.}
        \item {Yorkie 프로젝트의 메인테이너 및 멘토로 활동 중.}
        \item {Yorkie 프로젝트 전반적으로 다양한 기능, 최적화, 이슈 등을 제안.}
      \end{cvitems}
    }

%---------------------------------------------------------
  \cventry
  {멤버} % Affiliation/role
  {Kubernetes} % Organization/group
  {kubernetes-sig-docs} % Location
  {2023.04 - } % Date(s)
  {
    \begin{cvitems} % Description(s) of experience/contributions/knowledge
      \item {Kubernetes는 컨테이너화된 워크로드와 서비스를 관리하기 위한 이식성이 있고, 확장 가능한 오픈소스 플랫폼.}
      \item {kubernetes-sig-docs 한글화 팀에서 번역 및 문서 업데이트를 진행.}
      \item {kubernetes v1.26 버전 기준 8건의 신규 문서 번역 진행 및 outdated files 업데이트 진행.}
    \end{cvitems}
  }

%---------------------------------------------------------
  \cventry
    {멤버} % Affiliation/role
    {Istio} % Organization/group
    {Istio Community} % Location
    {2023.04 - } % Date(s)
    {
      \begin{cvitems} % Description(s) of experience/contributions/knowledge
        \item {Istio는 서비스 메시를 위한 오픈소스 플랫폼.}
        \item {Istio 커뮤니티의 멤버로 활동.}
        \item {istio.io 문서에 누락된 gRPC probe health checking과 관련하여 문서 업데이트를 수행.}
      \end{cvitems}
    }

%---------------------------------------------------------
\end{cventries}
