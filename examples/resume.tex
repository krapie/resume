%!TEX TS-program = xelatex
%!TEX encoding = UTF-8 Unicode
% Awesome CV LaTeX Template for CV/Resume
%
% This template has been downloaded from:
% https://github.com/posquit0/Awesome-CV
%
% Author:
% Kevin Park <krapi0314@gmail.com>
%
% Template license:
% CC BY-SA 4.0 (https://creativecommons.org/licenses/by-sa/4.0/)
%


%-------------------------------------------------------------------------------
% CONFIGURATIONS
%-------------------------------------------------------------------------------
% A4 paper size by default, use 'letterpaper' for US letter
\documentclass[11pt, a4paper]{awesome-cv}
% Use Korean fonts
\usepackage{kotex}

% Configure page margins with geometry
\geometry{left=1cm, top=.8cm, right=1cm, bottom=1.8cm, footskip=.5cm}

% Color for highlights
% Awesome Colors: awesome-emerald, awesome-skyblue, awesome-red, awesome-pink, awesome-orange
%                 awesome-nephritis, awesome-concrete, awesome-darknight
\colorlet{awesome}{awesome-darknight}
% Uncomment if you would like to specify your own color
% \definecolor{awesome}{HTML}{3E6D9C}

% Colors for text
% Uncomment if you would like to specify your own color
% \definecolor{darktext}{HTML}{414141}
% \definecolor{text}{HTML}{333333}
% \definecolor{graytext}{HTML}{5D5D5D}
% \definecolor{lighttext}{HTML}{999999}
% \definecolor{sectiondivider}{HTML}{5D5D5D}

% Set false if you don't want to highlight section with awesome color
\setbool{acvSectionColorHighlight}{true}

% If you would like to change the social information separator from a pipe (|) to something else
\renewcommand{\acvHeaderSocialSep}{\quad\textbar\quad}


%-------------------------------------------------------------------------------
%	PERSONAL INFORMATION
%	Comment any of the lines below if they are not required
%-------------------------------------------------------------------------------
% Available options: circle|rectangle,edge/noedge,left/right
\photo[circle,edge,right]{./examples/profile}
\name{박지환}{}
\position{데브옵스 엔지니어{\enskip\cdotp\enskip}백엔드 엔지니어}
\address{대한민국, 서울특별시 성동구 광나루로2길 34-13, 405}

\mobile{(+82) 10-2783-5972}
\email{krapi0314@gmail.com}
%\dateofbirth{January 1st, 1970}
%\homepage{www.posquit0.com}
\github{krapie}
\linkedin{krapie}
% \gitlab{gitlab-id}
% \stackoverflow{SO-id}{SO-name}
% \twitter{@twit}
% \skype{skype-id}
% \reddit{reddit-id}
% \medium{madium-id}
% \kaggle{kaggle-id}
% \googlescholar{googlescholar-id}{name-to-display}
%% \firstname and \lastname will be used
% \googlescholar{googlescholar-id}{}
% \extrainfo{extra information}

% \quote{백엔드, 인프라, 오픈소스, 클라우드에 관심이 많은 개발자입니다.}


%-------------------------------------------------------------------------------
\begin{document}

% Print the header with above personal information
% Give optional argument to change alignment(C: center, L: left, R: right)
\makecvheader[C]

% Print the footer with 3 arguments(<left>, <center>, <right>)
% Leave any of these blank if they are not needed
\makecvfooter
  {\today}
  {박지환~~~·~~~이력서}
  {\thepage}


%-------------------------------------------------------------------------------
%	CV/RESUME CONTENT
%	Each section is imported separately, open each file in turn to modify content
%-------------------------------------------------------------------------------
% %-------------------------------------------------------------------------------
%	SECTION TITLE
%-------------------------------------------------------------------------------
\cvsection{Summary}


%-------------------------------------------------------------------------------
%	CONTENT
%-------------------------------------------------------------------------------
\begin{cvparagraph}

%---------------------------------------------------------
I'm a cloud engineer at Amazon Web Services (AWS). And I'm interested in overall infrastructure, architecture, cloud, devops, distributed systems and so on.

I'm also an open source maintainer of Yorkie and a member of Kubernetes and Istio, and have many other open source activities. And I love to share my knowledge through contributing and mentoring.
\end{cvparagraph}

%-------------------------------------------------------------------------------
%	SECTION TITLE
%-------------------------------------------------------------------------------
\cvsection{Skills}


%-------------------------------------------------------------------------------
%	CONTENT
%-------------------------------------------------------------------------------
\begin{cvskills}

%---------------------------------------------------------
  \cvskill
    {DevOps} % Category
    {AWS, Docker, Kubernetes, Istio, Helm, ArgoCD, Jenkins, Terraform} % Skills

%---------------------------------------------------------
  \cvskill
    {Back-end} % Category
    {Spring Framework, REST API, MySQL, JPA, gRPC, MongoDB} % Skills

%---------------------------------------------------------
  \cvskill
    {Front-end} % Category
    {React, TailWindCSS} % Skills

%---------------------------------------------------------
  \cvskill
    {Programming} % Category
    {Go, JAVA, C++, YAML} % Skills

%---------------------------------------------------------
  \cvskill
    {Languages} % Category
    {Korean, English} % Skills

%---------------------------------------------------------
\end{cvskills}

%-------------------------------------------------------------------------------
%	SECTION TITLE
%-------------------------------------------------------------------------------
\cvsection{Experience}


%-------------------------------------------------------------------------------
%	CONTENT
%-------------------------------------------------------------------------------
\begin{cventries}

%---------------------------------------------------------
  \cventry
    {데브옵스 엔지니어 (인턴)} % Job title
    {당근페이 주식회사} % Organization
    {대한민국, 서울} % Location
    {2023.06 - 2023.08} % Date(s)
    {
      \begin{cvitems} % Description(s) of tasks/responsibilities
        \item {당근페이의 배포 생산성을 높이기 위한 데브옵스 도구 개발.}
        \item {신규 서비스의 성능 테스트 수행.}
        \item {핀테크 데브옵스/SRE 업무 수행.}
      \end{cvitems}
    }

%---------------------------------------------------------
  \cventry
    {백엔드 엔지니어 (계약)} % Job title
    {네이버 주식회사} % Organization
    {대한민국, 서울} % Location
    {2023.02 - 2023.06} % Date(s)
    {
      \begin{cvitems} % Description(s) of tasks/responsibilities
        \item {실시간 협업 애플리케이션 위한 오픈소스 문서 저장소 "Yorkie"의 프로덕션 환경 지원을 위한 Sharded 클러스터 모드 디자인 및 개발.}
        \item {Kubernetes와 Istio를 구현체로 Sharded 클러스터 모드 구현 및 Helm Chart로 패키징/배포.}
        \item {Yorkie의 프로덕션 환경 지원을 위해 Yorkie 서버의 master/slave 기반 GC, Document 삭제 API 등을 구현.}        
        \item {Terraform, ArgoCD, Prometheus, Grafana, Loki를 이용하여 Yorkie SaaS 데브옵스 시스템을 재구축.}
        \item {새로운 CRDT 자료구조 방식인 yorkie.tree를 Go SDK에 포팅.}
      \end{cvitems}
    }

%---------------------------------------------------------
  \cventry
    {오픈 프런티어 개발자 \& 멘토} % Job title
    {오픈 프런티어 (정보통신산업진흥원 주관 및 과학기술정보통신부 주최)} % Organization
    {Seoul, S.Korea} % Location
    {Mar. 2023 - Nov. 2023} % Date(s)
    {
      \begin{cvitems} % Description(s) of tasks/responsibilities
        \item {2023년 오픈 프런티어 개발자로 Yorkie 프로젝트 및 기타 오픈소스 활동 수행.}
        \item {2023년 오픈소스 플레이그라운드의 Yorkie 프로젝트 멘토로 활동.}
      \end{cvitems}
    }

%---------------------------------------------------------
  \cventry
    {소프트웨어 엔지니어} % Job title
    {소프트웨어 마에스트로 (정보통신기획평가원 주관 및 과학기술정보통신부 주최)} % Organization
    {대한민국, 서울} % Location
    {2022.03 - 2022.11} % Date(s)
    {
      \begin{cvitems} % Description(s) of tasks/responsibilities
        \item {시차가 존재하는 원격 팀을 위한 화이트보드 + 비디오 로그 기반 비동기 협업 소프트웨어 "Asyncrum" 기획 및 개발.}
        \item {Asyncrum 전체 소프트웨어 개발 프로세스 수립 및 짧은 개발 사이클을 가져가기 위한 스크럼 방식/도구 도입}.
        \item {백엔드 아키텍처 설계 및 AWS 클라우드 인프라 구축, 이후 v1.0부터 v1.6까지의 아키텍처 개선.}
        \item {Jenkins와 ArgoCD를 통한 백엔드 CI/CD 파이프라인 구축.}
        \item {화이트보드 오픈소스 tldraw와 동시 편집 오픈소스 Yorkie를 결합하여 실시간 협업 화이트보드 프로토타이핑 및 구현.}     
        \item {실시간 협업 화이트보드의 지연시간을 throttle 방식을 통해 대략 70\% 개선.}    
      \end{cvitems}
    }

%---------------------------------------------------------
  \cventry
    {소프트웨어 엔지니어} % Job title
    {PH2Y (세종대학교)} % Organization
    {대한민국, 서울} % Location
    {2021.07 - 2022.06} % Date(s)
    {
      \begin{cvitems} % Description(s) of tasks/responsibilities
        \item {안드로이드 지도 기반 반려동물 정보 공유 플랫폼 "집사의 노트" 서비스 기획 및 개발.}
        \item {자바 이미지 스케일링 라이브러리인 imgscalr를 활용한 백엔드 이미지 처리 서비스 구현 및 이미지 데이터 14배 \~ 72배 최적화.}
        \item {안드로이드 비디오 스트리밍을 위해 HTTP range request를 활용하여 비디오 스트리밍 서비스 구현.}
      \end{cvitems}
    }

%---------------------------------------------------------
  % \cventry
  %   {Software Engineer} % Job title
  %   {ShitOne Corp.} % Organization
  %   {Seoul, S.Korea} % Location
  %   {Dec. 2011 - Feb. 2012} % Date(s)
  %   {
  %     \begin{cvitems} % Description(s) of tasks/responsibilities
  %       \item {Developed a proxy drive smartphone application which connects proxy driver and customer.}
  %       \item {Implemented overall Android application logic and wrote API server for community service, along with lead engineer who designed bidding protocol on raw socket and implemented API server for bidding.}
  %     \end{cvitems}
  %   }

%---------------------------------------------------------
\end{cventries}

%-------------------------------------------------------------------------------
%	SECTION TITLE
%-------------------------------------------------------------------------------
\cvsection{Activity}


%-------------------------------------------------------------------------------
%	CONTENT
%-------------------------------------------------------------------------------
\begin{cventries}

%---------------------------------------------------------
  \cventry
    {메인테이너} % Affiliation/role
    {Yorkie} % Organization/group
    {Yorkie Team} % Location
    {2023.03 - } % Date(s)
    {
      \begin{cvitems} % Description(s) of experience/contributions/knowledge
        \item {Yorkie는 실시간 협업 애플리케이션을 개발하기 위한 오픈소스 문서 저장소.}
        \item {Yorkie 프로젝트의 메인테이너 및 개발자, 멘토로 활동 중.}
        \item {Yorkie 프로젝트 전반적인 기능, 최적화, 이슈 등을 제안.}
      \end{cvitems}
    }

%---------------------------------------------------------
  \cventry
  {멤버} % Affiliation/role
  {Kubernetes \& Istio} % Organization/group
  {kubernetes-sig-docs \& Istio} % Location
  {2023.04 - } % Date(s)
  {
    \begin{cvitems} % Description(s) of experience/contributions/knowledge
      \item {Kubernetes는 컨테이너화된 워크로드와 서비스를 관리하기 위한 이식성이 있고, 확장 가능한 오픈소스 플랫폼.}
      \item {Istio는 서비스 메시를 위한 오픈소스 플랫폼.}
      \item {kubernetes-sig-docs 및 Istio 커뮤니티의 멤버로 활동.}
      \item {kubernetes.io 및 istio.io 문서 업데이트 및 번역 활동 수행.}
    \end{cvitems}
  }

%---------------------------------------------------------
\cventry
    {오픈 프런티어 개발자 \& 멘토} % Affiliation/role
    {오픈 프런티어 (정보통신산업진흥원 주관 및 과학기술정보통신부 주최)} % Organization/group
    {대한민국, 서울} % Location
    {2023.05 - } % Date(s)
    {
      \begin{cvitems} % Description(s) of experience/contributions/knowledge
        \item {2024년 오픈소스 컨트리뷰션 아카데미의 Yorkie 프로젝트 리드 멘토 수행.}
        \item {2023년 오픈소스 컨트리뷰션 아카데미의 Yorkie 프로젝트 부 멘토 수행.}
        \item {2023년 오픈 프런티어 개발자로 Yorkie 프로젝트 개발 및 확산 활동 수행.}
      \end{cvitems}
    }
    
%---------------------------------------------------------
  \cventry
    {멘티 \& 엑스퍼트} % Affiliation/role
    {소프트웨어 마에스트로 (정보통신기획평가원 주관 및 과학기술정보통신부 주최)} % Organization
    {대한민국, 서울} % Location
    {2022.03 - 2024.06} % Date(s)
    {
      \begin{cvitems} % Description(s) of experience/contributions/knowledge
        \item {소프트웨어 마에스트로에서 13기 연수생 및 15기 엑스퍼트로 활동.}
        \item {13기 멘티로 시차가 존재하는 원격 팀을 위한 화이트보드 + 비디오 로그 기반 비동기 협업 소프트웨어 "Asyncrum" 기획 및 개발.}
        \item {15기 엑스퍼트로 15기 연수생들을 대상으로 멘토링 진행.}
      \end{cvitems}
    }

%---------------------------------------------------------
\end{cventries}

% \input{resume/honors.tex}
% %-------------------------------------------------------------------------------
%	SECTION TITLE
%-------------------------------------------------------------------------------
\cvsection{Certificates}


%-------------------------------------------------------------------------------
%	CONTENT
%-------------------------------------------------------------------------------
\begin{cvhonors}

%---------------------------------------------------------
  \cvhonor
    {AWS Certified Solutions Architect – Associate} % Name
    {AWS} % Issuer
    {ab1120546c...} % Credential ID
    {2024.09} % Date(s)

%---------------------------------------------------------
  \cvhonor
    {Certified Kubernetes Administrator (CKA)} % Name
    {CNCF} % Issuer
    {LF-91asjxo9jn} % Credential ID
    {2024.02} % Date(s)

%---------------------------------------------------------
  \cvhonor
    {Certified Kubernetes Application Developer (CKAD)} % Name
    {CNCF} % Issuer
    {LF-Iwyfq17542} % Credential ID
    {2024.02} % Date(s)

%---------------------------------------------------------
\end{cvhonors}

% \input{resume/presentation.tex}
% \input{resume/writing.tex}
% \input{resume/committees.tex}
%-------------------------------------------------------------------------------
%	SECTION TITLE
%-------------------------------------------------------------------------------
\cvsection{Education}


%-------------------------------------------------------------------------------
%	CONTENT
%-------------------------------------------------------------------------------
\begin{cventries}

%---------------------------------------------------------
  \cventry
    {컴퓨터공학과 학사} % Degree
    {세종대학교} % Institution
    {대한민국, 서울} % Location
    {2018.03 - } % Date(s)
    {
      \begin{cvitems} % Description(s) bullet points
        % \item {}
      \end{cvitems}
    }

%---------------------------------------------------------
\end{cventries}



%-------------------------------------------------------------------------------
\end{document}
